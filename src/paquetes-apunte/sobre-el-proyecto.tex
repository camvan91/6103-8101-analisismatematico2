\documentclass[apunte.tex]{subfiles}
\begin{document}
\newpage
\newcommand{\sobreelproyectosectionname}{Sobre el proyecto}
\section*{\sobreelproyectosectionname}
\markboth{}{\uppercase{\sobreelproyectosectionname}}
\addcontentsline{toc}{section}{\sobreelproyectosectionname}
FIUBA Apuntes nació con el objetivo de digitalizar los apuntes de las materias 
que andan rondando por los pasillos de FIUBA.

Además, queremos que cualquier persona sea libre de usarlos, corregirlos y mejorarlos.

Encontrarás más información sobre el proyecto o más apuntes en 
\href{http://fiuba-apuntes.github.io}{fiuba-apuntes.github.io}.

\newcommand{\usamoslatexsubsectionname}{¿Por qué usamos LaTeX?}
\subsection*{\usamoslatexsubsectionname}
%\addcontentsline{toc}{subsection}{\usamoslatexsubsectionname}
LaTeX es un sistema de composición de textos que genera documentos con alta 
calidad tipográfica, posibilidad de representación de ecuaciones y fórmulas 
matemáticas. Su enfoque es centrarse exclusivamente en el contenido sin tener 
que preocuparse demasiado en el formato.

LaTeX es libre, por lo que existen multitud de utilidades y herramientas para 
su uso, se dispone de mucha documentación que ayuda al enriquecimiento del 
estilo final del documento sin demasiado esfuerzo.

Esta herramienta es muy utilizado en el ámbito científico, para la publicación 
de papers, tesis u otros documentos. Incluso, en FIUBA, es utilizado para crear 
los enunciados de exámenes y apuntes oficiales de algunos cursos.

\newcommand{\usamosgitsubsectionname}{¿Por qué usamos Git?}
\subsection*{\usamosgitsubsectionname}
%\addcontentsline{toc}{subsection}{\usamosgitsubsectionname}
Git es un software de control de versiones de archivos de código fuente desde 
el cual cualquiera puede obtener una copia de un repositorio, poder realizar 
aportes tanto realizando \textit{commits} o como realizando \textit{forks} para 
ser unidos al repositorio principal.

Su uso es relativamente sencillo y su filosofía colaborativa permite que se 
sumen colaboradores a un proyecto fácilmente.

GitHub es una plataforma que, además de ofrecer los repositorios git, ofrece 
funcionalidades adicionales muy interesantes como gestor de reporte de errores.

\newpage
\end{document}